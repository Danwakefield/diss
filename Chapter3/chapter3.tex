\chapter{Implementation}

%The implementation should look at any issues you encountered as you tried to implement your design. During the work, you might have found that elements of your design were unnecessary or overly complex; perhaps third party libraries were available that simplified some of the functions that you intended to implement. If things were easier in some areas, then how did you adapt your project to take account of your findings?

%It is more likely that things were more complex than you first thought. In particular, were there any problems or difficulties that you found during implementation that you had to address? Did such problems simply delay you or were they more significant? 

%You can conclude this section by reviewing the end of the implementation stage against the planned requirements. 

\section{Arithmetic}
The arithmetic construct was the first thing I created.
It is a standalone piece of code only exposing a single method, Parse, which takes a string representing an equation and returning the integer value it evaluates too or an error.
Because of this isolation I created it as a sub package which also allowed me to simplify the error handling.
I was able to panic and recover\footnote{Panic is similar to an exception in other languages but has different semantics.Errors should be passed through the use of multiple return values instead.} to completely unwind the parser and lexer.
Although panics are reserved for truly exceptional cases in Go, they had to be used in this case.
Errors encountered during lexing or parsing of a language are almost always fatal as they leave the system in an indeterminate state.
For example:
\begin{verbatim}

\end{verbatim}

The recovered panic was then turned into an error and passed back as the return value as is idiomatic in Go.
Panicking across package boundaries is   



, apart from the fact that it can get and set variables.



Most of my research was done during this stage as I was still unsure on the type of lexer or parser that would best suit my project.
Once 

The lexer is simple as symbols are either digits, variable names --- which always start with a letter or underscore --- or identifiers which are special characters like '+'.




\subsection{Lexer}


\subsection{Parser}

\subsection{Variables}
\subsection{Ternary Bug}

\section{Scope}

\subsection{Variables}
\subsection{User Functions}
\subsection{Aliases}

\section{Main Lexer}

\subsection{Strings}
\subsection{Variables}
\subsection{Subshells}
\subsection{Arithmetic}

\section{Main Parser, AST \& Nodes}

\subsection{Simple Command}
\subsection{Command}
\subsection{Pipeline}
\subsection{And Or}
\subsection{List}

\section{Expansion}

\subsection{Tilde}
\subsection{Substitutions}
\subsection{Globbing}
\subsection{Word Splitting}

\section{Builtin Commands}

\section{Circular Imports}




