\chapter{Implementation}

%The implementation should look at any issues you encountered as you tried to implement your design. During the work, you might have found that elements of your design were unnecessary or overly complex; perhaps third party libraries were available that simplified some of the functions that you intended to implement. If things were easier in some areas, then how did you adapt your project to take account of your findings?

%It is more likely that things were more complex than you first thought. In particular, were there any problems or difficulties that you found during implementation that you had to address? Did such problems simply delay you or were they more significant? 

%You can conclude this section by reviewing the end of the implementation stage against the planned requirements. 

\section{Arithmetic Lexer \& Parser}
This was the first component that was created.
The fact that it stands alone and its only interactions are getting and setting variables made it the perfect starting point.

I
Most of my research was done during this stage as I was still unsure on the type of lexer or parser that would best suit my project.
Once 

The lexer is simple as symbols are either digits, variable names --- which always start with a letter or underscore --- or identifiers which are special characters like '+'.





\subsection{Variables}
\subsection{Ternary Bug}

\section{Scope}

\subsection{Variables}
\subsection{User Functions}
\subsection{Aliases}

\section{Main Lexer}

\subsection{Strings}
\subsection{Variables}
\subsection{Subshells}
\subsection{Arithmetic}

\section{Main Parser, AST \& Nodes}

\subsection{Simple Command}
\subsection{Command}
\subsection{Pipeline}
\subsection{And Or}
\subsection{List}

\section{Expansion}

\subsection{Tilde}
\subsection{Substitutions}
\subsection{Globbing}
\subsection{Word Splitting}

\section{Builtin Commands}

\section{Circular Imports}




