\section{Expansion}
Expansions are performed throughout the code to different degrees.
Some places dont require all of the possible expansions and explicitly pass flags to disable them.
This is a change from the C expansion code where the use of bitflags for every call was required.
I decided that the change was needed as it is rare to not want and expansion and it is easier to understand the logic of
'NoExpandTilde' than a series of or'ed flags.

Expansions are evaluated every time they are reached because of their 'impure' nature.

\subsection{Tilde}
When a tilde occurs either as the first character of a word or immediately following a ':' in a word it should be expanded.
The '\textasciitilde' can also be followed by text to indicate it should expand to the home directory of another system user.
Some shells have extended support for this and allow setting 'named directories' that can be expanded in the same way.

Support for this in the shell is basic.
If the first character is a tilde it is replaced by the path of the current users home directory.
The location of this directory comes from the 'os/user' package which wraps the platform dependant code needed to get this.
POSIX requires that the 'HOME' variable is checked before a lookup is conducted but the shell does not currently do this.

\subsection{Substitutions}
\label{sec:substitution}
Substitutions affect variables, commands and arithmetic.
Currently only commands can contain embedded substitutions but in the future all three will be able to nest each other.
This nesting means that it is possible to reach

XXX
\subsection{Globbing}
Globbing is the act of expanding the shells simple builtin pattern matching language to a list of files.
This look similar to regex syntax but isnt.
'*' can match any amount of any character, '?' any single character, and '[a-z]' a single character from the range.

By default glob looks in the current directory for files matching the provided pattern.
However if the pattern starts with a path that is the directory that will be searched.

I did not find a fnmatch function when I searched Go's documentation as so began looking online.
I eventually came across a years old gist that I forked, updated and used for this task.
It was only later that I realised that Go does provide the functionality I was looking for in the 'path/filepath' package.
Since the fnmatch library is only used in a few places and it is a single function call it offers no benefit over using the standard library function and should be swapped at the first opportunity. 

\subsection{Word Splitting}
Word splitting uses the value of a variable called the Internal Field Separator(IFS) to split expansions into multiple words.
The default value of this variable is ' \\n\\t' which splits on spaces, tabs and newlines.
While this may seem sensible it can lead to confusing behaviour and it is commonly changed to '\\n\\t'
for a so called strict mode.

This usually only occurs when variable and command substitutions are unquoted but some variables are special and have different behaviour.
E.g When the '\$*' variable occurs quoted it uses the first character of IFS to join the shells positional parameter into a single string.
\begin{verbatim}
IFS='-'
set - a b c
echo $@    # "a" "b" "c"
echo $*    # "a" "b" "c"
echo "$@"  # "a" "b" "c"
echo "$*"  # "a-b-c"
\end{verbatim}

Gosh does not perform word splitting of any kind.
This is probably the biggest problem with the implementation as it is required for correctly passing arguments to commands.
It is also used with special variables and in argument lists of functions and for loops.




