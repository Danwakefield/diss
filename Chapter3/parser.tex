\section{Main Parser, AST \& Nodes}
The parser turns the tokens into nodes and constructs AST's with them.
It was developed iteratively as other elements of the code where completed. XXX

\subsection{Simple Command}
A simple command is a line that contains a command with optional arguments and variable assignments.
The command can be a builtin (See~\ref{sec:builtins}), a function (See~\ref{sec:user-functions}) or an external command.

If the line with assignments also has an external command call, the assignments are temporary and are removed when the command completes.

During the third iteration this was created to support commands with arguments.
It was expanded during the fourth to accommodate variables assignments and during the fifth, for arguments that contained substitutions.
The simple command function is also the place that function definitions are detected and this was added during the penultimate iteration.

\subsection{Command}
Command is the function that parses the command structures of the code.
This includes the 'if', 'for', 'until', 'while' and 'case' expressions.
The code for parsing these is separated from the main function as they can be quite long.
One of the longest functions in the codebase is the parsing routine for a case statement and by splitting these up it allows you to focus on one thing at a time.

Along with the commands that people expect this is also the place that command grouping occurs.
This feature is often seen when defining shell functions and some people may believe that, like in other languages, it is a part of their definition syntax.
However it can be used anywhere a command can including in a pipeline or in a conditional to avoid certain bugs.
A command group begins with a '\{' and ends with a corresponding '\}'.
Anything contained between the two is parsed as a separate group which allows redirections to be applied to everything it contains.
See the code in~\ref{lst:command-grouping} for examples.

\subsection{Pipeline}
The pipeline is a list of commands that will join their inputs and outputs to each other and then be run in parallel.

This allows simple utilities to be combined to produce complex transformations.
Since they are conducted in memory buffers overhead can be reduced by such an amount that they can be faster than common distributed computing tools like Hadoop\cite{AdamD45:online}.

Pipelines encourage the Unix philosophy of programs that 'Do one thing and do it well'.
Text output is almost universal 



\subsection{And Or}
\subsection{List}

