\section{Builtin Commands}
\label{sec:builtins}
Builtin commands are used for two reasons, they either modify internal state and as such cannot be replaced by external commands or they are used often enough that the startup overhead of the executables being called could affect performance

or they are builtin to reduce the performance costs of starting up new executables for commonly used commands.

In the last iteration some of the most needed built-in commands were developed.
This included 'cd' and 'local' which have to be built-in because they modify internal state in the Scope struct. 

I also implemented the 'true' and 'false' for performance reasons.
While these are not bottlenecks in any program they show that commands can be easily created.
The biggest boosts will probably come from implementing 'echo' and 'printf' as these are commonly used commands.

As these commands are logically separate from the functionality of the shell I created a subpackage for them.
This again showed that package planning had not been considered well.
Both IOContainer and ExitStatus had to be moved from 'package main' to a subpackage.
This package has ended up with the nondescript name of 'package T' as it has no purpose other than to be a temporary home to these types.

XXX