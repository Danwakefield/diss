\chapter{Testing}

%Detailed descriptions of every test case are definitely not what is required here. What is important is to show that you adopted a sensible strategy that was, in principle, capable of testing the system adequately even if you did not have the time to test the system fully.

%Have you tested your system on "real users"? For example, if your system is supposed to solve a problem for a business, then it would be appropriate to present your approach to involve the users in the testing process and to record the results that you obtained. Depending on the level of detail, it is likely that you would put any detailed results in an appendix.

%the following sections indicate some areas you might include. Other sections may be more appropriate to your project. 

\section{Overall Approach to Testing}

\subsection{Unit Tests}

\subsection{User Interface Testing}

\subsection{Stress Testing}

\section{Automated Testing}
Towards the end of the project I started using a continuous integration service called drone.io\footnote{My project's public page is available at \url{https://drone.io/github.com/Danwakefield/gosh}}.
Every time I pushed a commit to my github repository it would clone the code and run the both the unit and integration tests against the changes.
A notification email was sent if a build failed.

The system just ran the normal unit and integration tests however since this codebase can issue system calls it is safer to run these tests in a disposable container in case of a severe bug.  
CI systems also make it easy for contributors to see the status of the code and ensure that pull requests will not cause any regression.

\subsection{Other types of testing}

\section{Integration Testing}

\section{User Testing}