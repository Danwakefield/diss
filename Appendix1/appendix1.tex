\chapter{Third-Party Code and Libraries}

%If you have made use of any third party code or software libraries, i.e. any code that you have not designed and written yourself, then you must include this appendix. 

%As has been said in lectures, it is acceptable and likely that you will make use of third-party code and software libraries. The key requirement is that we understand what is your original work and what work is based on that of other people. 

%Therefore, you need to clearly state what you have used and where the original material can be found. Also, if you have made any changes to the original versions, you must explain what you have changed. 

%As an example, you might include a definition such as: 

%Apache POI library Ð The project has been used to read and write Microsoft Excel files (XLS) as part of the interaction with the clientÕs existing system for processing data. Version 3.10-FINAL was used. The library is open source and it is available from the Apache Software Foundation 
%. The library is released using the Apache License 
%. This library was used without modification. 

\lstinputlisting[caption={[fnmatch]fnmatch function. Originally from a gist published in 2010 \url{https://gist.github.com/kballard/272720}. See my fork of the code at \url{https://github.com/Danwakefield/fnmatch}. Forking was required as the code was created before Go V1 and I had to correct behaviour that had changed in that time. Attempted contact with the author failed\label{lst:fnmatch}}]{Appendix1/fnmatch.go}

XXX Move somewhere it counts. 
While Go has an fnmatch implementation available (import "path/filepath") is is hardcoded with settings for matching against file names. 
This would not work in the case list where globbing should not expand to a list of filenames but still requires matching against the same pattern syntax.
Therefore external code to perform this task was included.
LINK Github + gist.

