\chapter{Evaluation}

%Examiners expect to find in your dissertation a section addressing such questions as:

%\begin{itemize}
%   \item Were the requirements correctly identified? 
%   \item Were the design decisions correct?
%   \item Could a more suitable set of tools have been chosen?
%   \item How well did the software meet the needs of those who were expecting to use it?
%   \item How well were any other project aims achieved?
%   \item If you were starting again, what would you do differently?
%\end{itemize}

%Such material is regarded as an important part of the dissertation; it should demonstrate that you are capable not only of carrying out a piece of work but also of thinking critically about how you did it and how you might have done it better. This is seen as an important part of an honours degree. 

%There will be good things and room for improvement with any project. As you write this section, identify and discuss the parts of the work that went well and also consider ways in which the work could be improved. 

%Review the discussion on the Evaluation section from the lectures. A recording is available on Blackboard. 

\section{Requirements}
XXX


\section{Design Decisions}
As has been noted in~\ref{sec:implementation}, a few of the design decisions where clearly wrong. This section aims to show the problems and talk about possible future solutions to them.

\subsection{Lexer}
The biggest problem was the design of the lexer being non re-entrant.
This prevented many syntax features from being completed and has left the shell in a non compliant state.
This things that where affected include:
\begin{itemize*}
	\item Embedded variables in both the arithmetic construct and complex substitutions.
    \item Complex substitutions that take multiple arguments.
    \item Alias expansions.
\end{itemize*}

While substitutions and embedded variables could have been accomplished with sublexers this would have been quite inefficient. 
Each lexer currently requires a complete copy of the remaining input text, a problem when parsing large scripts.
Switching the lexers to take a Bytes.Buffer instead of a string for the input may help as slices\footnote{An array like structure that uses pointers into an actual array to reduce overhead.} could be taken from it.

Alias expansion is more complex as the input text can be modified considerably.
As they occur completely in the lexer they can introduce any text including multi-line strings or control structures.
This places strain on parts of the code that require knowledge of a tokens position in the source.
Error messages are particularly affected as if they indicate that problems occur on lines that don't exist or don't contain the error it would be very confusing. 

With a lightweight sublexer aliases could be expanded in a separate context and control returned to after the replacement in the original lexer.
Tokens could be modified to indicate how many alias replacements have occurred before they are emitted and the original line that the alias was expanded at.
This would require careful and complex bookkeeping however as aliases can be expanded multiple times.
Other shells also include the ability for expansions happen in places other that the first word of a command which would make it even more complex.

\subsection{Package Layout}
Another of the bad decisions was the lack of planning in package layout.



\subsection{a}


\section{Suitability of tools}

\section{Other project aims}

\section{What would you do differently.}













