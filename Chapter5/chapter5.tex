\chapter{Evaluation}

%Examiners expect to find in your dissertation a section addressing such questions as:

%\begin{itemize}
%   \item Were the requirements correctly identified? 
%   \item Were the design decisions correct?
%   \item Could a more suitable set of tools have been chosen?
%   \item How well did the software meet the needs of those who were expecting to use it?
%   \item How well were any other project aims achieved?
%   \item If you were starting again, what would you do differently?
%\end{itemize}

%Such material is regarded as an important part of the dissertation; it should demonstrate that you are capable not only of carrying out a piece of work but also of thinking critically about how you did it and how you might have done it better. This is seen as an important part of an honours degree. 

%There will be good things and room for improvement with any project. As you write this section, identify and discuss the parts of the work that went well and also consider ways in which the work could be improved. 

%Review the discussion on the Evaluation section from the lectures. A recording is available on Blackboard. 

\section{Requirements}
As the requirements come mostly from the specification, it is hard 


\section{Design Decisions}
As has been noted in~\ref{sec:implentation}, a few of the design decisions where clearly  wrong.

The biggest problem was the design of the lexer being non re-entrant.
This prevented many syntax features from being completed and has left the shell in a non compliant state.
This things that where affected include:
\begin{itemize*}
	\item Embedded variables in both the arithmetic construct and complex substitutions.
    \item Complex substitutions that take multiple arguments.
    \item Alias expansions.
\end{itemize*}



\section{Suitability of tools}

\section{Other project aims}

\section{What would you do differently.}













